% 使用自定义文档类
\documentclass{../pkslide}




% =============================================================================
% 以下是导言区



% ===================================================================
% 调用宏包


% \usepackage{relsize}


% 用于添加 CC BY-SA 4.0 版权协议
\usepackage[
  type={CC},
  modifier={by-sa},
  version={4.0},
  lang={chinese-utf8}
]{doclicense}
\renewcommand{\doclicenseLongText}{%
  本作品采用\space\href{https://creativecommons.org/licenses/by-sa/4.0/deed.zh}{Creative Commons“署名-相同方式共享 4.0 国际”}许可协议进行许可。\xspace%
}



% ===================================================================
% 自定义一些命令


% C++ 符号宏,依赖 hyperref 和 relsize
% 来自 GitHub 上的仓库 tcbrindle/wg21papertemplate 中的 ./common.tex 文件
\newcommand{\Cpp}{\texorpdfstring{C\kern-0.05em\protect\raisebox{.35ex}{\textsmaller[2]{+\kern-0.05em+}}}{C++}}


% 用于数学公式中的自定义命令
\newcommand{\nequiv}{\not\equiv} % 不同余
\newcommand{\lcm}{\operatorname{lcm}} % 最小公倍数
\newcommand{\lpd}{\operatorname{lpd}} % 最小素因数



% ===================================================================
% 不影响显示形式的,需要在导言区引入的信息


% 标题、作者、学校/机构与日期信息,用于标题页显示和页脚显示:页脚将会显示短形式
\title[算法竞赛中的数论(格式测试)]{算法竞赛中的数论}
\subtitle{格式测试}
\firstauthor{%
  陈亮舟%
  %\inst{1}%
}[PinkRabbit]
\firstinstitute{%
  %\inst{1}%
  福建师范大学附属中学%
}[The Affiliated High School of Fujian Normal University]{AHSFNU}
\originalfinaldate*{2023 年 2 月 4 日}

% 如果需要引入更多作者或学校/机构(注意控制文字高度):
% \addauthor{忆艾\inst{2}}[Elegia]
% \addinstitute{\inst{2}艾兰学院,理论计算机科学学院}[Department of Theoretical Computer Science\\AiLan Academy]{AiLan}

% 一些其他格式为:
% \firstauthor{陈亮舟}
% \firstauthor{陈亮舟}[PinkRabbit][PK]
% \firstinstitute{清华大学}{THU}


% 添加关于信息,也可不添加
\aboutinfotext{%
  本系列课件的 PDF 版本和 \LaTeX{} 源代码均托管于\href{https://github.com/GitPinkRabbit/Number-Theory-in-Competitive-Programming}{“算法竞赛中的数论 -- 系列课件”-- GitHub/GitPinkRabbit},后续更新将在此上传。
  
  \doclicenseThis%
}



% 以上是导言区
% =============================================================================




% =============================================================================
% 正文自此开始


\begin{document}
\begin{frame}
  \frametitle{中文演示文档}
  \begin{itemize}
    \item 你需要将所有源文件保存为 UTF-8 编码
    \item 你可以使用 \XeLaTeX{}、\LuaLaTeX{} 或 \upLaTeX{} 编译
    \item 也可以使用 \pdfLaTeX{} 编译
    \item 推荐使用 \XeLaTeX{} 或 \LuaLaTeX{} 编译
    \item 对高级用户,我们也推荐使用 \upLaTeX{} 编译
  \end{itemize}
  
  此页面来自《\CTeX{} 宏集手册》。
  
  事实上,应统一使用 \XeLaTeX{} 编译本系列课件。
  
  使用 \emph{latexmk} 可以自动处理 {\sffamily beamer} 可能需要的多次编译流程。
  
  使用 rm 族 \textrm{\XeLaTeX} 和 tt 族 \texttt{\XeLaTeX}。
\end{frame}

\begin{frame}
  \frametitle{含有公式的页面}
  Leonhard Euler 于 1734 年解决了 Basel 问题:$\sum_{i = 1}^{+\infty} \frac{1}{i^2} = \frac{\pi^2}{6}$,或
  
  \[ \sum_{i = 1}^{+\infty} \frac{1}{i^2} = \frac{\pi^2}{6} \text{。} \]
  
  这里使用一些由 \AmSLaTeX{} 提供的数学公式:
  
  \[ \sum_{i = 0}^{n} \binom{n}{i} = 2^n \text{。} \]
  
  使用 \texttt{\textbackslash{}mathbb}、\textcolor{red}{\texttt{\textbackslash{}mathsc}}、\texttt{\textbackslash{}mathcal} 和 \texttt{\textbackslash{}mathscr}:
  
  $\mathbb{R} \subset \mathbb{C}$、$\mathsc{Mset}(\mathcal{A})$ 和 $\mathscr{W} \cdot \mathscr{K}$。
\end{frame}

\begin{frame}
  \frametitle{beamer 版面设置}
  beamer 使用的是进行调整过后的 Berlin 主题。具体调整为:
  
  将页脚(footline)更换为 infolines 外部主题的形式。
  
  让页眉增加了显示 subsubsection 名的位置。
\end{frame}

\begin{frame}[fragile]
  \frametitle{显示代码}
  要使用 \verb|\verb|、\verb|verbatim| 环境、或 \verb|listings|、\verb|minted| 宏包等工具显示代码,需要在 \verb|\begin{frame}| 后添加 \verb|fragile| 选项。下面展示使用原生 \verb|verbatim| 环境显示 \Cpp{} 代码:
  
\begin{verbatim}
int gcd(int a, int b) {
  return b ? gcd(b, a % b) : a;
}
\end{verbatim}
  
  如上显示了函数 \verb|gcd|。
\end{frame}

\begin{frame}
  \frametitle{英文字体测试}
  每行依次为 rm、sf、tt 族。每三行依次为 up、it、sl 形。后一半使用 bf 加粗。
  
  {%
    {\upshape {\rmfamily Ab-cd--ffig, hmw.} {\sffamily Ab-cd--ffig, hmw.} {\ttfamily Ab-cd--ffig, hmw.}}
    
    {\itshape {\rmfamily Ab-cd--ffig, hmw.} {\sffamily Ab-cd--ffig, hmw.} {\ttfamily Ab-cd--ffig, hmw.}}
    
    {\slshape {\rmfamily Ab-cd--ffig, hmw.} {\sffamily Ab-cd--ffig, hmw.} {\ttfamily Ab-cd--ffig, hmw.}}
  }%
  
  {\bfseries%
    {\upshape {\rmfamily Ab-cd--ffig, hmw.} {\sffamily Ab-cd--ffig, hmw.} {\ttfamily Ab-cd--ffig, hmw.}}
    
    {\itshape {\rmfamily Ab-cd--ffig, hmw.} {\sffamily Ab-cd--ffig, hmw.} {\ttfamily Ab-cd--ffig, hmw.}}
    
    {\slshape {\rmfamily Ab-cd--ffig, hmw.} {\sffamily Ab-cd--ffig, hmw.} {\ttfamily Ab-cd--ffig, hmw.}}
  }%
  
  我更换了英文 tt 族字体变为 \texttt{Cascadia Mono}。sl 形似只对 rm 族起作用。
\end{frame}

\begin{frame}
  \frametitle{中文字体测试}
  每行依次为 rm、sf、tt 族。每两行依次为 up、it 形。后一半使用 bf 加粗。
  
  {%
    {\upshape {\rmfamily{}我能吞下玻璃而不伤身体。}{\sffamily{}我能吞下玻璃而不伤身体。}{\ttfamily{}我能吞下玻璃而不伤身体。}}
    
    {\itshape {\rmfamily{}我能吞下玻璃而不伤身体。}{\sffamily{}我能吞下玻璃而不伤身体。}{\ttfamily{}我能吞下玻璃而不伤身体。}}
    
    {\bfseries%
      {\upshape {\rmfamily{}我能吞下玻璃而不伤身体。}{\sffamily{}我能吞下玻璃而不伤身体。}{\ttfamily{}我能吞下玻璃而不伤身体。}}
      
      {\itshape {\rmfamily{}我能吞下玻璃而不伤身体。}{\sffamily{}我能吞下玻璃而不伤身体。}{\ttfamily{}我能吞下玻璃而不伤身体。}}
    }%
  }%
  
  我对于 rm、sf、tt 族选用的中文字体分别是“\textrm{思源宋体}”、“\textsf{思源黑体}”、“\texttt{方正黑仿\ 简}”,it 形统一使用“\textit{霞鹜文楷}”。
\end{frame}

\begin{frame}
  \frametitle{其他拉丁文字字体测试}
  每行依次为 rm、sf、tt 族。每两行依次为 up、it、sl 形。后一页使用 bf 加粗。
  
  拉丁字母:
  
  {%
    \only<1>{%
      {\upshape {\rmfamily François Viète, Gauß, Pǎtraşcu.} {\sffamily François Viète, Gauß, Pǎtraşcu.} {\ttfamily François Viète, Gauß, Pǎtraşcu.}}
      
      {\itshape {\rmfamily François Viète, Gauß, Pǎtraşcu.} {\sffamily François Viète, Gauß, Pǎtraşcu.} {\ttfamily François Viète, Gauß, Pǎtraşcu.}}
      
      {\slshape {\rmfamily François Viète, Gauß, Pǎtraşcu.} {\sffamily François Viète, Gauß, Pǎtraşcu.} {\ttfamily François Viète, Gauß, Pǎtraşcu.}}
    }%
    
    \only<2>{\bfseries%
      {\upshape {\rmfamily François Viète, Gauß, Pǎtraşcu.} {\sffamily François Viète, Gauß, Pǎtraşcu.} {\ttfamily François Viète, Gauß, Pǎtraşcu.}}
      
      {\itshape {\rmfamily François Viète, Gauß, Pǎtraşcu.} {\sffamily François Viète, Gauß, Pǎtraşcu.} {\ttfamily François Viète, Gauß, Pǎtraşcu.}}
      
      {\slshape {\rmfamily François Viète, Gauß, Pǎtraşcu.} {\sffamily François Viète, Gauß, Pǎtraşcu.} {\ttfamily François Viète, Gauß, Pǎtraşcu.}}
    }%
  }%
  
  由于字体原因,不打算直接使用希腊字母和西里尔字母,有需求应使用拉丁文转写。
\end{frame}

\begin{frame}[c]
  \frametitle{一页简单的定理环境与动画测试}
  \only<5>{此例子改编自 beamer 用户手册。注意前后文间距。}
  
  \begin{theorem}[可选定理名]
    不存在最大的素数。
  \end{theorem}
  
  \begin{proof}[可选证明附加文字]
    \begin{enumerate}
      \item<1-> 假设 $p$ 为最大的素数。
      \item<3-> Let $q$ be the product of the first $p$ numbers. 以下跳一步。
      \item<2-> But $q + 1$ is greater than $1$, thus divisible by some prime
      number not in the first $p$ numbers.\qedhere
    \end{enumerate}
  \end{proof}
  
  \only<4>{此例子改编自 beamer 用户手册。本页特意使用纵向居中。}
\end{frame}

\begin{frame}[c]
  \frametitle{“定义”环境}
  前文。注意环境与前后文间距。
  
  \begin{definition}[素数]
    恰好有 $2$ 个正因数的正整数称为{素数(质数,prime number)}。
    
    即,唯二的两个正因数为 $1$ 和它本身。
  \end{definition}
  
  后文。本页特意使用纵向居中。
\end{frame}

\begin{frame}[fragile]
  \frametitle{测试自定义命令}
  第一行。以下使用 \verb|\emptyline| 空一行。
  
  \emptyline
  
  第二行。
  
  测试数学公式:$\lcm(10, 15) = 30$,$\lpd(45) = 3$。
  \begin{align*}
    \lcm(10, 15) &= 30 \\
    \lpd(45) &= 3
  \end{align*}
  {\tiny
    在 \verb|\tiny| 下。第一行。以下使用 \verb|\emptyline| 空一行。
    
    \emptyline
    
    第二行。\verb|\baselineskip|${}={}$\the\baselineskip
    
    \vspace{0.6\baselineskip}
    
    第三行。
    
  } % 注意前面的换行,这确保最后有文字的行一定完整包含在 \tiny 内
\end{frame}

\begin{frame}[c]
  \frametitle{以下是文档结构测试}
  确保不报 overfull vbox。不跨级使用 subsubsection(包括此文件)。
  
  \begin{multicols}{2}
    \tableofcontents
  \end{multicols}
\end{frame}

\section{第一节}

\begin{frame}
  \frametitle{空着的一页}
\end{frame}

\begin{frame}
  \frametitle{空着的第二页}
  \framesubtitle{我是空着的(吗?)}
  算了还是写点东西吧?
  \[ \sum_{i = 1}^{n} \mu(i) > 0 \text{。} \]
\end{frame}

\subsection{第一节的第一小节}

\begin{frame}
  \frametitle{空着的一页}
\end{frame}

\begin{frame}
  \frametitle{空着的第二页}
  \framesubtitle{我是空着的(吗?)}
  算了还是写点东西吧?
  \[ \sum_{i = 1}^{n} \mu(i) > 0 \text{。} \]
\end{frame}

\subsubsection{更小的小节}

\begin{frame}
  \frametitle{空着的一页}
\end{frame}

\begin{frame}
  \frametitle{空着的第二页}
  \framesubtitle{我是空着的(吗?)}
  算了还是写点东西吧?
  \[ \sum_{i = 1}^{n} \mu(i) > 0 \text{。} \]
\end{frame}

\subsubsection{更小的小节二}

\begin{frame}
  \frametitle{空着的一页}
\end{frame}

\begin{frame}
  \frametitle{空着的第二页}
  \framesubtitle{我是空着的(吗?)}
  算了还是写点东西吧?
  \[ \sum_{i = 1}^{n} \mu(i) > 0 \text{。} \]
\end{frame}

\subsection{第一节的第二小节}

\begin{frame}
  \frametitle{空着的一页}
\end{frame}

\begin{frame}
  \frametitle{空着的第二页}
  \framesubtitle{我是空着的(吗?)}
  算了还是写点东西吧?
  \[ \sum_{i = 1}^{n} \mu(i) > 0 \text{。} \]
\end{frame}

\subsubsection{更小的小节三}

\begin{frame}
  \frametitle{空着的一页}
\end{frame}

\begin{frame}
  \frametitle{空着的第二页}
  \framesubtitle{我是空着的(吗?)}
  算了还是写点东西吧?
  \[ \sum_{i = 1}^{n} \mu(i) > 0 \text{。} \]
\end{frame}

\subsection{第一节的第三小节}

\begin{frame}
  \frametitle{空着的一页}
\end{frame}

\begin{frame}
  \frametitle{空着的第二页}
  \framesubtitle{我是空着的(吗?)}
  算了还是写点东西吧?
  \[ \sum_{i = 1}^{n} \mu(i) > 0 \text{。} \]
\end{frame}

\section{第二节}

\begin{frame}
  \frametitle{空着的一页}
\end{frame}

\begin{frame}
  \frametitle{空着的第二页}
  \framesubtitle{我是空着的(吗?)}
  算了还是写点东西吧?
  \[ \sum_{i = 1}^{n} \mu(i) > 0 \text{。} \]
\end{frame}

\subsection{第二节的第一小节}

\begin{frame}
  \frametitle{空着的一页}
\end{frame}

\begin{frame}
  \frametitle{空着的第二页}
  \framesubtitle{我是空着的(吗?)}
  算了还是写点东西吧?
  \[ \sum_{i = 1}^{n} \mu(i) > 0 \text{。} \]
\end{frame}

\subsection{第二节的第二小节}

\begin{frame}
  \frametitle{空着的一页}
\end{frame}

\begin{frame}
  \frametitle{空着的第二页}
  \framesubtitle{我是空着的(吗?)}
  算了还是写点东西吧?
  \[ \sum_{i = 1}^{n} \mu(i) > 0 \text{。} \]
\end{frame}

\subsubsection{更小的小节四}

\begin{frame}
  \frametitle{空着的一页}
\end{frame}

\begin{frame}
  \frametitle{空着的第二页}
  \framesubtitle{我是空着的(吗?)}
  算了还是写点东西吧?
  \[ \sum_{i = 1}^{n} \mu(i) > 0 \text{。} \]
\end{frame}

\section{第三节}

\begin{frame}
  \frametitle{空着的一页}
\end{frame}

\begin{frame}
  \frametitle{空着的第二页}
  \framesubtitle{我是空着的(吗?)}
  算了还是写点东西吧?
  \[ \sum_{i = 1}^{n} \mu(i) > 0 \text{。} \]
\end{frame}

\section{第四节}

\begin{frame}
  \frametitle{空着的一页}
\end{frame}

\begin{frame}
  \frametitle{空着的第二页}
  \framesubtitle{我是空着的(吗?)}
  算了还是写点东西吧?
  \[ \sum_{i = 1}^{n} \mu(i) > 0 \text{。} \]
\end{frame}

\subsection{第四节的第一小节}
\subsubsection{更小的小节五}

\begin{frame}
  \frametitle{空着的一页}
\end{frame}

\begin{frame}
  \frametitle{空着的第二页}
  \framesubtitle{我是空着的(吗?)}
  算了还是写点东西吧?
  \[ \sum_{i = 1}^{n} \mu(i) > 0 \text{。} \]
\end{frame}

\section*{不编号的第五节}

\begin{frame}
  \frametitle{空着的一页}
\end{frame}

\begin{frame}
  \frametitle{空着的第二页}
  \framesubtitle{我是空着的(吗?)}
  算了还是写点东西吧?
  \[ \sum_{i = 1}^{n} \mu(i) > 0 \text{。} \]
\end{frame}

\subsection{第五节的第一小节}

\begin{frame}
  \frametitle{空着的一页}
\end{frame}

\begin{frame}
  \frametitle{空着的第二页}
  \framesubtitle{我是空着的(吗?)}
  算了还是写点东西吧?
  \[ \sum_{i = 1}^{n} \mu(i) > 0 \text{。} \]
\end{frame}

\subsection{第五节的第二小节}

\begin{frame}
  \frametitle{空着的一页}
\end{frame}

\begin{frame}
  \frametitle{空着的第二页}
  \framesubtitle{我是空着的(吗?)}
  算了还是写点东西吧?
  \[ \sum_{i = 1}^{n} \mu(i) > 0 \text{。} \]
\end{frame}

\section{第六节}

\subsection{第六节的第一小节}

\begin{frame}
  \frametitle{空着的一页}
\end{frame}

\begin{frame}
  \frametitle{空着的第二页}
  \framesubtitle{我是空着的(吗?)}
  算了还是写点东西吧?
  \[ \sum_{i = 1}^{n} \mu(i) > 0 \text{。} \]
\end{frame}

\subsection*{不编号的第六节的第二小节}

\begin{frame}
  \frametitle{空着的一页}
\end{frame}

\begin{frame}
  \frametitle{空着的第二页}
  \framesubtitle{我是空着的(吗?)}
  算了还是写点东西吧?
  \[ \sum_{i = 1}^{n} \mu(i) > 0 \text{。} \]
\end{frame}

\subsection{第六节的第三小节}

\begin{frame}
  \frametitle{空着的一页}
\end{frame}

\begin{frame}
  \frametitle{空着的第二页}
  \framesubtitle{我是空着的(吗?)}
  算了还是写点东西吧?
  \[ \sum_{i = 1}^{n} \mu(i) > 0 \text{。} \]
\end{frame}
\end{document}


% 正文自此结束
% =============================================================================
