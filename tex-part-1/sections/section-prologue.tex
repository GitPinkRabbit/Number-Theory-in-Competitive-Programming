% !TeX root = ../NTCP-Part-1.tex

\begin{frame}
  \frametitle{导言}
  对每个人来说,自降生以来,认识到的第一个数学对象,都是自然数。
  \pause
  
  由自然数间的四则运算开始,数论(number theory)的研究自然而然地产生了。
  \pause
  
  \emptyline
  即使是最基础的数论成果,也足以在算法竞赛实践中取得丰富成果。
  \pause
  
  \emptyline
  本册课件介绍数论的基础知识,配以在算法竞赛中的应用,既是“开胃小菜”,也试图增进初学者在一些细节上的理解,希望能够帮助读者建立“数论在算法竞赛中的理论与实践”的方法论上的基本认识,为后面的内容做好铺垫。
\end{frame}
\begin{frame}
  \frametitle{前置知识}
  作为本系列课件主线的肇始,本册需要的前置知识非常少:
  \begin{mymulticols}[l][l]{2}
    \begin{itemize}
      \item 义务教育阶段一至六年级数学
      \begin{itemize}
        \item 自然数的四则运算、有余数的除法
        \item 小数、分数、负数的运算
        \item 因数与倍数、素数与合数
        \item 长方体的体积公式
        \item \textboldcolor[orange]{其中与数论相关的内容会在本册第一章进行复习}
      \end{itemize}
      \item 义务教育阶段七至八年级数学的\textboldcolor[orange]{部分内容},如下
      \begin{itemize}
        \item 平方根、立方根
        \item 函数、坐标系、函数图像
        \item 幂的运算、整式的运算
        \item 注意:\textboldcolor[orange]{平面几何部分不做要求}
        \item \alert{如果缺少这些知识,可以先阅读本系列课件的第零册}
      \end{itemize}
    \end{itemize}
  \end{mymulticols}
  \begin{itemize}
    \item 进制,特别是二进制
    \item \Cpp{} 基础语法、算法和时空复杂度的概念
    \item 一定的几何直观,或者说空间想象能力
  \end{itemize}
\end{frame}
