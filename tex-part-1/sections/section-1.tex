% !TeX root = ../NTCP-Part-1.tex

\section{前置知识复习}
\begin{frame}
  \frametitle{前置知识复习}
  在正式开始前,让我们先对一些前置知识进行“特别复习”。请回忆:
  \begin{itemize}
    \item 有余数的除法
    \item 因数与倍数
    \item 素数与合数
  \end{itemize}
  \pause
  
  \emptyline
  除了帮助读者回忆起相关概念,这也是为了在本系列中明确\textbf{这些基础概念的特殊情况}的具体定义。
  
  有一些概念在不同文献中的定义不同,特别是对于特殊情况的处理。对于有争议的定义,在阅读本系列时,请以本系列给出的定义为准。
  \pause
  
  \emptyline
  对于不需要复习的读者,可以只注意本节中使用\textboldcolor[darkgreen]{绿色粗体字}标注的部分,明确对于特殊情况的定义。
\end{frame}
\subsection{有余数的除法}
\begin{frame}[c]
  \progressnow*
\end{frame}
\begin{frame}
  \frametitle{回忆与定义}
  回忆:对于整数 $a$ 和正整数 $m$,可以计算 $a$ 除以 $m$,得到商 $q$ 和余数 $r$。
  
  对此,有熟悉的记号:“\textboldcolor[orange]{$a \div m = q \cdots \cdots r$}”。若 $r = 0$,可以省略后部。
  
  \emptyline
  下面我们给出有余数的除法的\textbf{精确定义}。
  \pause
  \begin{definition}[有余数的除法]
    对于整数 $a$ 和正整数 $m$,\textboldcolor[orange]{存在且仅存在}一对整数 $\langle q, r \rangle$ 满足
    \begin{center}
      $a = q \cdot m + r$ 且 $0 \le r < m$,
    \end{center}
    \pause
    则称 $a$ 除以 $m$ 的\fdf{商}为 $q$、\fdf{余数}为 $r$。并称其中的 $a$ 为\fdf{被除数}、$m$ 为\fdf{除数}。
  \end{definition}
  \pause
  请注意,被除数可以为\textboldcolor[darkgreen]{任意\CJKunderdot[format=\color{darkgreen}]{整数}},而除数限定为\textcolor{darkgreen}{\textbf{\CJKunderdot[format=\color{darkgreen}]{正}整数}}。
\end{frame}
\begin{frame}
  \frametitle{例子}
  \begin{itemize}
    \item $5$ 除以 $3$ 的商为 \fillblank<2->[red]{$1$},余数为 \fillblank<2->[red]{$2$}。
    \pause
    \pause
    \item $-9$ 除以 $5$ 的商为 \fillblank<4->[darkgreen]{$-2$},余数为 \fillblank<4->[darkgreen]{$1$}。
  \end{itemize}
  \pause
  \pause
  请特别注意,\textboldcolor[darkgreen]{当 $a$ 为负整数时,同样认为 $a$ 除以 $m$ 的余数 $r$ 是非负数}。
  
  而此时商 $q$ 仍要满足 $q \cdot m = a - r$。
  \pause
  \begin{itemize}
    \item $-12$ 除以 $4$ 的商为 \fillblank<7->[red]{$-3$},余数为 \fillblank<7->[red]{$0$}。
    \pause
    \pause
    \item ${10}^6$ 除以 $7$ 的商为 \fillblank<9->[red]{$142857$},余数为 \fillblank<9->[red]{$1$}。
  \end{itemize}
\end{frame}
\begin{frame}
  \frametitle{向下取整}
  向下取整符号“$\lfloor {\cdot} \rfloor$”:对于实数 $x$,定义 $\lfloor x \rfloor$ 为小于等于 $x$ 的最大整数。
  
  例如 $\lfloor 3.8 \rfloor = 3$、$\lfloor -4.2 \rfloor = -5$。整数向下取整后不变。
  \pause
  
  \emptyline
  如果使用向下取整符号,则有 $q = \bigl\lfloor \frac{a}{m} \bigr\rfloor$。
  \pause
  
  结合 $r = a - q \cdot m$,得到
  \[
    r = a - \Bigl\lfloor \frac{a}{m} \Bigr\rfloor \cdot m \text{。}
  \]
  \pause
  
  \PKMissing{这里需要一张图片}
\end{frame}
\subsubsection{\texorpdfstring{\textsuperscript{(*) }}{(*) }\Cpp{} 中整型的 \texttt{/} 和 \texttt{\%} 运算符}
\begin{frame}[c]
  \progressnow
\end{frame}
\begin{frame}
  \frametitle{\Cpp{} 中\textbf{内置类型}的 \texttt{/} 和 \texttt{\%} 运算符}
  在 \Cpp{} 中,提供了\alert{除法运算符}“\texttt{/}”和\alert{求余运算符}“\texttt{\%}”以支持 \Cpp{} 程序进行相关计算。它们的形式为:
  \begin{itemize}
    \item “\texttt{\textcolor{darkgray}{左操作数} / \textcolor{darkgray}{右操作数}}”,
    \item “\texttt{\textcolor{darkgray}{左操作数} \% \textcolor{darkgray}{右操作数}}”。
  \end{itemize}
  \pause
  
  \emptyline
  众所周知,\Cpp{} 中有两类\textbf{内置}的数值类型,分别是\alert{整数类型}和\alert{浮点类型}。
  \begin{itemize}
    \item 整型的例子:\texttt{int}、\texttt{long long} 与它们的无符号版本、以及 \texttt{char}。
    \item 浮点型的例子:\texttt{float}、\texttt{double}、以及 \texttt{long double}。
  \end{itemize}
  \pause
  其中,浮点型之间可以进行 \texttt{/} 运算,但\textboldcolor[orange]{不能}进行 \texttt{\%} 运算。
  
  当整型与浮点型之间进行运算时,会将整型操作数转换为浮点型。
  \pause
  
  \emptyline
  当然,我们关注的是整型之间的 \texttt{/} 和 \texttt{\%} 运算。
\end{frame}
\begin{frame}
  \frametitle{整型的 \texttt{/} 运算的舍入方向}
  从 \Cpp{}11 标准开始,规定了 \texttt{/} 运算符的舍入方向:运算结果为第一操作数除以第二操作数的数值结果(一个小数)\textboldcolor[orange]{向零舍入}得到的整数。
  \pause
  
  \emptyline
  换句话说,运算结果\alert{在绝对值上}总是小于等于数值结果。
  
  更具体地,如果数值结果 $> 0$,则运算结果 $\le$ 数值结果,\\
  \-\hspace{7em}数值结果 $< 0$,则运算结果 $\ge$ 数值结果,\\
  \-\hspace{7em}数值结果 $= 0$,则运算结果 $=$ 数值结果 $= 0$。
  \pause
  
  \emptyline
  与前文“有余数的除法”不同,\Cpp{} 中 \texttt{/} 和 \texttt{\%} 的\alert{第二操作数均可以为负}。
  \pause
  
  \emptyline
  特别地,如果第二操作数为 $0$,则行为未定义(UB)。典型的编译器实现可能导致运行时错误(RE)。
\end{frame}
\begin{frame}
  \frametitle{整型的 \texttt{\%} 运算}
  \Cpp{} 保证了:若 $a \mathbin{\texttt{\%}} m = r$ 以及 $a \mathbin{\texttt{/}} m = q$,则一定有 $a = q \mathbin{\texttt{*}} m \mathbin{\texttt{+}} r$。
  
  特别地,若 $a \mathbin{\texttt{/}} m$ 会触发未定义行为,则 $a \mathbin{\texttt{\%}} m$ 的行为也未定义。
  \pause
  
  \emptyline
  对比前文“有余数的除法”与 \Cpp{} 中整型的 \texttt{/} 和 \texttt{\%} 运算符,异同之处在于:
  \begin{itemize}
    \item 相同点:
    \begin{visibleenv}<3->
      \begin{itemize}
        \item 当 $a$ 是非负数且 $m$ 是正数时,两者的结果相同。
        \item 即使 $a$ 是负数,当 $m$ 是正数且 \textboldcolor[orange]{$a$ 是 $m$ 的倍数}时,两者的结果也相同。
        \item 当 $m = 0$ 时,两者都是“未定义”的。
      \end{itemize}
    \end{visibleenv}
    \item 不同点:
    \begin{visibleenv}<4->
      \begin{itemize}
        \item 当 $m$ 是负数时,“有余数的除法”没有定义,但在 \Cpp{} 中有定义。
        \item 当 $m$ 是正数且 $a$ 是负数时,记 $a \div m = r_1$ 而 $a \mathbin{\texttt{/}} m = r_2$,则 $r_2$ 要么与 $r_1$ 相等,要么比 $r_1$ 大 $1$。
      \end{itemize}
    \end{visibleenv}
  \end{itemize}
\end{frame}
\begin{frame}
  \frametitle{例子}
  \begin{itemize}
    \item \texttt{ 5 / \ 3} 与 \texttt{ 5 \% \ 3} 的结果分别为 \fillblank*<2->[red]{\texttt{ 1}} 与 \fillblank*<2->[red]{\texttt{ 2}}。
    \pause
    \pause
    \item \texttt{ 5 / -3} 与 \texttt{ 5 \% -3} 的结果分别为 \fillblank*<4->[red]{\texttt{-1}} 与 \fillblank*<4->[red]{\texttt{ 2}}。
    \pause
    \item \texttt{-5 / \ 3} 与 \texttt{-5 \% \ 3} 的结果分别为 \fillblank*<5->[red]{\texttt{-1}} 与 \fillblank*<5->[red]{\texttt{-2}}。
    \pause
    \item \texttt{-5 / -3} 与 \texttt{-5 \% -3} 的结果分别为 \fillblank*<6->[red]{\texttt{ 1}} 与 \fillblank*<6->[red]{\texttt{-2}}。
    \pause
    \pause
    \item \texttt{-7 / \ 0} 与 \texttt{-7 \% \ 0} 都是 \fillblank<8->[red]{未定义行为}。
    \pause
    \item \texttt{ 0 / \ 0} 与 \texttt{ 0 \% \ 0} 都是 \fillblank<9->[red]{未定义行为}。
  \end{itemize}
\end{frame}
\begin{frame}
  \frametitle{正负性提示}
  假设 $a \mathbin{\texttt{/}} m = q$ 和 $a \mathbin{\texttt{\%}} m = r$。
  
  则有 $q$ 的正负性与 $a \cdot m$ 的正负性相同,\\
  \-\hspace{1em}而 $r$ 的正负性与 $a$ 的正负性相同。
  
  \emptyline
  在操作数可能为负数或 $0$ 时,需要特别注意。
  \pause
  
  \emptyline
  为了身心健康,避免记忆乱七八糟的情况,良好的代码实践是:
  \begin{center}
    \textboldcolor[darkgreen]{尽量保证第一操作数为\textboldcolor[orange]{非负整数}、第二操作数为正整数。}
  \end{center}
  \pause
  
  在 \Cpp{} 代码实现上,除此之外,还有其他注意事项,请看下页。
\end{frame}
\begin{frame}
  \frametitle{\textbf{代码注意事项}:整数提升和整数转换}
  在 \Cpp{} 中,整型之间的算术运算将经过两道关卡:\alert{整数提升}和\alert{整数转换}。
\end{frame}
\begin{frame}
  \frametitle{\textbf{代码注意事项}:整数提升}
  可以认为\alert{整数提升}是,将“小于 \texttt{int}”的类型转换到 \texttt{int} 或 \texttt{unsigned int}。例如对 \texttt{bool}、\texttt{char}、\texttt{short} 或它们的有或无符号版本进行转换。特殊情况按下不表。\textboldcolor[orange]{整数提升\CJKunderdot[format=\color{orange}]{总是保留原数值不变}。整型算术运算\CJKunderdot[format=\color{orange}]{总是}先将两操作数进行整数提升。}
  
  \emptyline
  例如,在 \texttt{char} 之间的减法,实际上会转换为 \texttt{int} 进行。
  
  \emptyline
  如果从不进行小于 \texttt{int} 的类型的算术运算,则无需担心整数提升。
\end{frame}
\begin{frame}
  \frametitle{\textbf{代码注意事项}:整数转换}
  \alert{整数转换}可能对数值进行更改。具体地,进行整型算术运算时,会产生被称作“\alert{公共类型}”的结果类型,同时两操作数也会\textbf{先转换为公共类型}。
  
  \emptyline
  问题在于:当一操作数为有符号,另一操作数为无符号,且无符号操作数的“等级”大于或等于有符号操作数的“等级”时,产生的公共类型就是\textboldcolor[orange]{无符号操作数的类型}。此时将有符号操作数转换为公共类型时,可能发生数值更改。
  \pause
  
  \emptyline
  例如,计算 \texttt{-4 \% 3u} 时,产生的公共类型为 \texttt{unsigned int},此时先将 \texttt{-4} 转换得到 \texttt{4294967292u},进行除法得到结果 \texttt{1431655764u},与期望的 \texttt{-1} 不同。
  \pause
  同样地,良好的实践是:
  \begin{center}
    \textboldcolor[darkgreen]{尽量避免使用无符号类型进行 \texttt{/} 或 \texttt{\%} 运算。}
  \end{center}
\end{frame}
\subsection{整除性、因数与倍数}
\begin{frame}[c]
  \progressnow*
\end{frame}
\begin{frame}
  \frametitle{回忆与定义}
  回忆:在有余数的除法中,如果余数为 $0$,就说“\textboldcolor[orange]{\textit{被除数是除数的倍数}}”。
  \pause
  \begin{definition}[整除、因数与倍数]
    对于整数 $a, b$,且 $a \ne 0$,如果 \textboldcolor[orange]{$\frac{b}{a}$ 也是整数},则称 \fdf{$b$ 被 $a$ 整除},或称 \fdf{$a$ 整除 $b$},记作 \fdf{$a \mid b$},否则称 $a$ 不整除 $b$,记作 \fdf{$a \nmid b$}。
   
    \emptyline
    此时也称 $b$ 是 \fdf{$a$ 的倍数},$a$ 是 \fdf{$b$ 的因数(约数)}。
    \pause
    
    \emptyline
    同时,我们\alert{约定} $0$ 整除 $0$,但不整除其他任何整数。
    
    即 $0 \mid 0$,但是对于所有 $n \ne 0$ 均有 $0 \nmid n$。
  \end{definition}
  \pause
  请注意,这里 \textboldcolor[darkgreen]{$a, b$ 均可以为任意整数}。第四节会揭示考虑 $0$ 的原因。
  \pause
  
  \emptyline
  同时,在计算上,可以利用余数是否为 $0$ 判断是否整除。
\end{frame}
\begin{frame}
  \frametitle{例子}
  下列哪些整除关系是成立的?成立的打“$\surd$”,不成立的打“$\times$”。
  \begin{enumerate}
    \item $\hphantom{-{}}2 \mid \hphantom{-{}}6$\hspace{1em}(\visible<2->{\makebox[15pt]{\color{darkgreen}$\surd$}})
    \item $-6 \mid \hphantom{-{}}6$\hspace{1em}(\visible<3->{\makebox[15pt]{\color{darkgreen}$\surd$}})
    \item $\hphantom{-{}}5 \mid \hphantom{-{}}0$\hspace{1em}(\visible<4->{\makebox[15pt]{\color{darkgreen}$\surd$}})
    \item $-4 \mid -6$\hspace{1em}(\visible<5->{\makebox[15pt]{\color{red}$\times$}})
    \item $\hphantom{-{}}0 \mid \hphantom{-{}}0$\hspace{1em}(\visible<6->{\makebox[15pt]{\color{darkgreen}$\surd$}})
    \item $\hphantom{-{}}0 \mid -7$\hspace{1em}(\visible<7->{\makebox[15pt]{\color{red}$\times$}})
  \end{enumerate}
  \pause\pause\pause\pause\pause\pause\pause
  \emptyline
  使用“因数与倍数”的语言,上面的每一条应该如何说?
\end{frame}
\begin{frame}
  \frametitle{简单规律}
  关于整除的一些简单规律:
  \pause
  \begin{enumerate}
    \item $1$ 和 $-1$ 整除所有整数。
    \pause
    \item 所有整数整除 $0$。
    \pause
    \item 对于\alert{正}整数 $a$,$a$ 是 $a$ 的最大因数,也是 $a$ 的最小\alert{正}倍数。
    
    并且,$-a$ 是 $a$ 的最小因数,也是 $a$ 的最大负倍数。
  \end{enumerate}
  \pause
  使用“因数与倍数”的语言,上面的前两条应该如何说?
  \pause
  
  \emptyline
  一般地,对于任意整数 $a$,一定有 $1, -1, a, -a$ 是 $a$ 的因数,\\
  \-\hspace{10em}\hphantom{ $a$}也一定有 $\hphantom{0, -{}}0, a, -a$ 是 $a$ 的倍数。
  
  对于非 $0$ 整数 $a$,称 $1, -1, a, -a$ 为 $a$ 的\fdf{平凡因数}。
\end{frame}
\begin{frame}
  \frametitle{习题}
  仅使用定义,试着证明关于整除性的这些性质(出现的字母都表示整数):
  \pause
  \begin{enumerate}
    \setlength{\itemsep}{2pt}
    \item $a \mid a$ 且 $a \mid (-a)$ 且 $(-a) \mid a$。
    \pause
    \item 如果 $a \mid b$ 且 $b \mid c$,那么 $a \mid c$。
    \pause
    \item 如果 $a \mid b$ 且 $b \ne 0$,那么 $\lvert a \rvert \le \lvert b \rvert$。
    \pause
    \item 如果 $a \mid b$ 且 $b \mid a$,那么 $a = \pm b$。
    \pause
    \item 如果 $a \mid b$,那么 $a \mid (k \cdot b)$ 且 $(k \cdot a) \mid (k \cdot b)$。
    \pause
    \item 如果 $d \mid a$ 且 $d \mid b$,那么 $d \mid (a \pm b)$。
  \end{enumerate}
  \pause
  进一步地,利用上面的性质,试着证明这个很有用的推论:
  \pause
  \begin{enumerate}
    \setcounter{enumi}{6}
    \item 对于 $n$ 个数 $a_1, \ldots, a_n$,如果 $d$ 整除每个 $a_i$,那么
    \begin{center}
      \textboldcolor[orange]{$d \mid (k_1 a_1 + k_2 a_2 + \cdots + k_n a_n)$}。
    \end{center}
  \end{enumerate}
\end{frame}
\begin{frame}
  \frametitle{习题(补充)}
  经过上面的习题,我们发现:
  \pause
  \begin{itemize}
    \item “整除”与“小于等于”较为类似。因为可以写“$a \mid a$”和“$a \mid b \mid c$”,甚至还有 $a \mid b \implies \lvert a \rvert \le \lvert b \rvert$ 的性质(除了 $0$)。但它们的不同之处在于,对任意两个数总有其中一个小于等于另一个,但有可能它们互不整除对方。
    \pause
    \item 在谈论整除的时候,互为相反数的 $a$ 与 $-a$ 是等价的,可以互换。
    \pause
    \item 如果 $d$ 是 $a_1, \ldots, a_n$ 每一个的因数,则 $d$ 也是 $a_1, \ldots, a_n$ 的任何\fdf{整系数线性组合}的因数。
    
    上文中,$(k_1 a_1 + k_2 a_2 + \cdots + k_n a_n)$ 就是 $a_{1 \sim n}$ 的一个整系数线性组合。
  \end{itemize}
\end{frame}
\begin{frame}
  \frametitle{因数分布}
  \PKMissing{别急}
\end{frame}
\subsection{素数与合数}
\begin{frame}[c]
  \progressnow*
\end{frame}
\frame{}% delete this ===============================================
