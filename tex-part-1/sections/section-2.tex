% !TeX root = ../NTCP-Part-1.tex

\section{基础知识}
\begin{frame}
  \frametitle{引入}
  本章将介绍两个基础概念:“\alert{同余}”和“\alert{最大公因数与最小公倍数}”。后续册次中有大量内容依赖于它们。
  \begin{center}
    \huge\fbox{\quad$\equiv \pmod{m}$\quad}
  \end{center}
  \begin{center}
    \huge\fbox{\quad$\gcd$, $\lcm$\quad}
  \end{center}
  \pause
  
  \emptyline
  本章将引导读者通过\textbf{形式化}的数学推导证明有关这些概念的诸多性质,同时也从\textbf{直观}的角度进行一定的解释。掌握这些能力对学习数论大有裨益。
\end{frame}
\begin{frame}
  \frametitle{引入 -- 同余}
  “同余”这一概念,类似于我们熟悉的“相等”的概念,它是后者在\textbf{模意义下}的类比。我们将使用两种方式(整除和余数)来定义同余,然后说明两种方式一致,并将证明同余是一种“等价关系”。
  \pause
  
  \emptyline
  第六章(取模计算入门)将广泛使用同余的概念。在接下来三册中,同余更是占据了学习内容的核心地位。
\end{frame}
\begin{frame}
  \frametitle{引入 -- 最大公因数与最小公倍数}
  最大公因数与最小公倍数是整除关系中的核心概念。
  \pause
  \begin{itemize}
    \item 它们性质丰富、对理解概念帮助很大(见第四章(整除结构))。
    \pause
    \item 并且,在应用层面上,它们广泛出现于模型转化后的题目中。
    \pause
    \item 同时,它们也具有计算上的优越性(见 3.3 节(辗转相除法))。
  \end{itemize}
  \pause
  
  \emptyline
  最大公因数联系、沟通了数论中的诸多方面,所以在后续册次中,最大公因数将频繁出现。
\end{frame}
\subsection{同余}
\begin{frame}[c]
  \progressnow*
\end{frame}
\begin{frame}{别急}\PKMissing{别急}\end{frame} % delete this ==================
\subsection{最大公因数与最小公倍数}
\begin{frame}[c]
  \progressnow*
\end{frame}
\begin{frame}{别急}\PKMissing{别急}\end{frame} % delete this ==================
