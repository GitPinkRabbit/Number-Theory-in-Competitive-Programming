% 使用自定义文档类
\documentclass{../pkslide}




% =============================================================================
% 以下是导言区



% ===================================================================
% 调用宏包


\usepackage{tikz}


\usepackage{minted}


\usepackage{changepage}


\usepackage{graphbox}


\usepackage{fontawesome}


% 用于添加 CC BY-SA 4.0 版权协议
\usepackage[
  type={CC},
  modifier={by-sa},
  version={4.0}
]{doclicense}
\renewcommand{\doclicenseLongText}{%
  本作品采用\space\href{https://creativecommons.org/licenses/by-sa/4.0/deed.zh}{Creative Commons“署名-相同方式共享 4.0 国际”}许可协议进行许可。\xspace%
}



% ===================================================================
% 自定义一些命令


% C++ 符号宏,依赖 hyperref 和 relsize
% 来自 GitHub 上的仓库 tcbrindle/wg21papertemplate 中的 ./common.tex 文件
\newcommand{\Cpp}{\texorpdfstring{C\kern-0.05em\protect\raisebox{.35ex}{\textsmaller[2]{+\kern-0.05em+}}}{C++}}


% 用于数学公式中的自定义命令
\newcommand{\nequiv}{\not\equiv} % 不同余
\newcommand{\lcm}{\operatorname{lcm}} % 最小公倍数



% ===================================================================
% 不影响显示形式的,需要在导言区引入的信息


% 标题、作者、学校/机构与日期信息,用于标题页显示和页脚显示:页脚将会显示短形式
\title[算法竞赛中的数论(二)]{算法竞赛中的数论}
\subtitle{\texorpdfstring{\textsuperscript{2.}}{2. }模 \texorpdfstring{$m$}{m} 剩余系的加法与乘法结构及其应用}
\firstauthor{%
  陈亮舟%
  %\inst{1}%
}[PinkRabbit]
\firstinstitute{%
  %\inst{1}%
  福建师范大学附属中学%
}[The Affiliated High School of Fujian Normal University]{AHSFNU}
\originalfinaldate{\today}


% 添加关于信息,也可不添加
\aboutinfotext{%
  本系列课件的 PDF 版本和 \LaTeX{} 源代码均托管于\href{https://github.com/GitPinkRabbit/Number-Theory-in-Competitive-Programming}{“算法竞赛中的数论 -- 系列课件”-- GitHub/GitPinkRabbit},后续更新将在此上传。
  
  \doclicenseThis%
}


% 以上是导言区
% =============================================================================

\begin{document}

\begin{frame}
  \frametitle{引入}
  本课件中我们聚焦于给定一个固定的模数 $m$ 并且一切运算都对 $m$ 取模时可能遇到的数论问题。此处的 $m$ 可以为任意正整数。
  
  未特别说明时,本课件中的一切同余式都将省略模 $m$ 的提示性后缀,即将 $\mathcal A \equiv \mathcal B \pmod m$ 简写作 $\mathcal A \equiv \mathcal B$。
  
  \pause
  在内容编排上,每一节的内容为:
  
  \begin{enumerate}
    \item 介绍剩余类与剩余系的概念,并介绍在剩余系内进行运算时的注意事项与快速幂算法。
    \item 结合课件一中的知识讨论剩余系的加法结构。
    \item 初步讨论剩余系的乘法结构并介绍 Euler 函数、Fermat 小定理、Euler 定理以及它们的应用等等。
    \item 进一步讨论剩余系的乘法结构,介绍阶、原根、指标等概念与相关的一些应用,介绍离散对数问题与大步小步算法。
  \end{enumerate}
\end{frame}

\section{模 \texorpdfstring{$m$}{m} 剩余类与剩余系}
\subsection{剩余类与剩余系}
\pseudosubsubsection

\begin{frame}
  \frametitle{同余等价}
  回顾课件一中对同余的定义:对于整数 $a, b$,称 $a \equiv b$,当且仅当 $m \mid (a - b)$。也就是存在整数 $k$ 使得 $k m = a - b$,即 $a = b + k m$。
  
  容易证明同余是一种等价关系,即满足自反性($a \equiv a$)、对称性(如果 $a \equiv b$ 则 $b \equiv a$)、与传递性(如果 $a \equiv b$ 且 $b \equiv c$ 则 $a \equiv c$)。
  
  \pause
  同余是一种等价关系说明了可以将全体整数划分为若干类,每个整数恰好属于其中一类,每一类中的整数间两两同余,不同类之间的整数间两两不同余。
  
  \pause
  \begin{example}[等价类]
    \begin{itemize}
      \item 当 $m = 1$ 时,全体整数为一个等价类,因为 $1$ 整除所有整数。
      \item 当 $m = 6$ 时,$\{ \ldots, -10, -4, 2, 8, 14, \ldots \}$ 为一个等价类,容易验证其中任意两数同余,并且其中任意数与其他整数不同余。
    \end{itemize}
  \end{example}
\end{frame}

\begin{frame}
  \frametitle{剩余类与剩余系}
  \begin{onlyenv}<1>
    由于 $a$ 与所有 $a + k m$($k$ 为整数)必然在同一个等价类中,而其他整数都与 $a$ 不在同一个等价类中,所以每个等价类都形如 $a$ 加上若干(正数、零、或负数)倍的 $m$。
    
    \begin{definition}[剩余类]
      对于整数 $a$,称所有同余于 $a$ 模 $m$ 的整数构成的集合为一个\fdf{模 $m$ 剩余类(residue class modulo $m$)},显然 $a$ 在它本身定义的剩余类中,这个剩余类形如 $\{ \ldots, a - 2 m, a - m, a, a + m, a + 2 m, \ldots \}$。
    \end{definition}
  \end{onlyenv}
  
  \pause
  \begin{onlyenv}<2-4>
    \begin{definition}[剩余系]
      恰好有 $m$ 个剩余类,它们分别可以由 $0, 1, \ldots, m - 1$ 定义得到。
      
      将 $m$ 个剩余类看作元素,它们构成的集合称为\fdf{模 $m$ 剩余系(residue system modulo $m$)}。
      
      \pause
      对于任意整数 $i$,将 $i$ 定义得到的剩余类直接记作 $i$,后文中不再使用记号区分整数与剩余类。可以发现同余的两数 $a, b$ 定义得到的剩余类相同,在此意义上可以写作 $a = b$。
    \end{definition}
    
    \pause
    \begin{example}[剩余系]
      当 $m = 6$ 时,$\{ 0, 1, 2, 3, 4, 5 \}$ 为剩余系,而 $\{ -2, -1, 3, 6, 7, 14 \}$ 也为剩余系,因为它们均表示 $m$ 个不同的剩余类,只是记号不同。
    \end{example}
  \end{onlyenv}
\end{frame}

\begin{frame}
  \frametitle{剩余类与剩余系 -- 小结}
  尽管此后整数与剩余类使用同一记号,但请注意模意义下剩余系仍然是一个抽象的概念。
  
  例如在剩余系中 $\underbrace{1 + 1 + \cdots + 1}_{m \text{ 个}} = 0$,这是通常逻辑无法想象的。
  
  再例如线性同余方程 $a x \equiv b \pmod{m}$ 的解集本可以表示为 $x \equiv x_0 \pmod{\frac{m}{\gcd(a, m)}}$,但在剩余类的语言中可写作 $x = x_0$(模 $\frac{m}{\gcd(a, m)}$),仅有唯一解。\\
  通过此例,可以发现,也需要注意在谈论剩余类与剩余系时明确模数,否则在一些情况下可能会造成混乱。
\end{frame}

\subsection{剩余类的运算}
\subsubsection{剩余类的运算}

\begin{frame}
  \frametitle{剩余类的运算}
  我们还未在剩余类间定义运算。好在,同余的性质保证了剩余类间的运算可以如整数运算一般自然地定义:对于两个剩余类 $x, y$,它们的和、差、积只需定义为分别从其中取出任意一个整数后,两个整数间的对应运算结果所在的剩余类。
    \pause
  可以证明无论怎样选取这两个整数,运算结果所在的剩余类均不会改变。
  
  \pause
  \begin{example}[剩余类的运算]
    当 $m = 6$ 时(总是将剩余类记作 $[0, m - 1]$ 中的整数):
    
    \pause
    \begin{itemize}
      \item 剩余类 $4, 5$ 的和可以为 $4 + 5 = 9$ 所在的剩余类,即 $3$。
        \pause
      \item 剩余类 $4, 5$ 的差可以为 $4 - (-1) = 5$ 所在的剩余类,即 $5$。
        \pause
      \item 剩余类 $4, 5$ 的积可以为 $(-2) \cdot (-1) = 2$ 所在的剩余类,即 $2$。
    \end{itemize}
    
    \pause
    即,表示剩余类时,$4 + 5 = 3$、$4 - 5 = 5$、$4 \cdot 5 = 2$。
  \end{example}
\end{frame}

\begin{frame}
  \frametitle{剩余类运算与整数同余式的关系}
  \begin{onlyenv}<1-4>
    在上例中,我们看到当 $m = 6$ 时有 $4 + 5 = 3$、$4 - 5 = 5$、$4 \cdot 5 = 2$。
    
    \uncover<2->{然而这与同余式如出一辙:同样有 $4 + 5 \equiv 3$、$4 - 5 \equiv 5$、$4 \cdot 5 \equiv 2$,这里将数字看作整数而非剩余类。}\uncover<3->{定义了剩余类之间的加法、减法、与乘法后,任何由剩余类、或与整数混合构成的多项式的运算结果都可以被合理定义,并且运算结果与同余式的运算结果将精确对应。}
    
    \uncover<4->{%
      \emptyline
      那么,剩余类,抛开与同余式的对应关系,的额外意义是什么呢?%
    }
  \end{onlyenv}
  
  \begin{onlyenv}<5-6>
    这里或需提到几点:
    
    \begin{itemize}
      \item 剩余类与整数的对应关系,在某种意义上并不精确。当 $m = 4$ 时,尽管作为剩余类有 $0 = 0 \cdot 2$,但这并不意味着整数 $4$ 能被两个分别在剩余类 $0, 2$ 中的整数相乘得到。若表示成集合内元素相乘,应有 $0 \cdot 2 = \{ \ldots, -16, -8, 0, 8, 16, \ldots \}$。
      \item 同余式两侧的整数运算结果是整数,保留了运算的一切信息,而剩余类只保留了余数信息。
      \item 在运算中,绝不是所有对象都可以转换为剩余类进行,例如幂运算中的指数仍应看作非负整数而非剩余类。
    \end{itemize}
    
    \begin{uncoverenv}<6>
      \emptyline
      算法竞赛实践中,使用剩余类时,既有可能指剩余类对象本身,又有可能指这一剩余类中的特定整数,还有可能指这一剩余类中的全体整数,需要留心辨别剩余类概念在上下文中的含义。
    \end{uncoverenv}
  \end{onlyenv}
\end{frame}

\subsubsection{计算机中的表示}

\begin{frame}
  \frametitle{剩余类在计算机中的表示}
  一共有 $m$ 个剩余类,即 $\{ 0, 1, \ldots, m - 1 \}$。在计算机中,它们可以直接使用整数类型进行表示:每个剩余类恰好自然地对应一个在 $[0, m - 1]$ 内的整数,即这个剩余类中唯一在 $[0, m - 1]$ 内的整数。
  
  \pause
  \begin{example}[剩余类在 $\lbrack 0, m - 1 \rbrack$ 内的表示]
    当 $m = 6$ 时,
    
    \begin{itemize}
      \item 剩余类 $3 = \{ \ldots, -9, -3, 3, 9, 15, \ldots \}$ 对应整数 $3$,
      \item 剩余类 $17 = \{ \ldots, 5, 11, 17, 23, 29, \ldots \}$ 对应整数 $5$。
    \end{itemize}
    
    \pause
    剩余类 $-1 = \{ \ldots, -1 - 2 m, -1 - m, -1, m - 1, 2 m - 1 \}$ 总是对应整数 $m - 1$。
  \end{example}
  
  \pause
  可以发现,即使给定整数,它定义的剩余类的表示并不一定是这个整数本身,但总与这个整数同余。
\end{frame}

\begin{frame}
  \frametitle{剩余类的运算在计算机中的表示}
  若严格遵循使用 $[0, m - 1]$ 内的整数表示剩余类,计算机执行算数运算时,可能导致运算结果偏离这个范围,尽管得到的结果仍然与正确表示同余,此时需要进行额外操作将结果拉回范围内。
  
  \begin{itemize}
    \item 将整数 $x$ 所在的剩余类转换为整数表示:$x \bmod m$。
    \item 将整数表示为 $x, y$ 的剩余类之和拉回范围内:$(x + y) \bmod m$。
    \item 将整数表示为 $x, y$ 的剩余类之差拉回范围内:$(x - y) \bmod m$。
    \item 将整数表示为 $x, y$ 的剩余类之积拉回范围内:$(x \cdot y) \bmod m$。
  \end{itemize}
  
  \pause
  \emptyline
  一般来说,必须假设剩余类和整数取模进行加法、减法、乘法运算时是 $\mathcal O (1)$ 的,后续课件中将探讨实践中遇到的算法常数问题。
\end{frame}

\begin{frame}
  \frametitle{剩余类的运算在计算机中的表示 -- 注意事项}
  然而,在 \Cpp{} 中,由于除法运算符向零舍入,使用 \texttt{\%} 表示取模时,对于中间结果可能为负的需要进行修正,例如减法应写为
  \begin{center}
    \texttt{($x$ - $y$ + $m$) \% $m$}。
  \end{center}
    \pause
  同时,需要考虑整型溢出的问题,这在乘法中尤为常见,因为常见的模数 $m$ 一般是在 $2^{30}$ 范围内的大数,而两个整数表示相乘得到的结果范围是 $[0, {(m - 1)}^2]$,只有当 $m \le 46341$ 时,在单次乘法中才没有可能出现整型溢出的情况。当 $m$ 可能 $> 46341$ 时,乘法应写为
  \begin{center}
    \texttt{((long long)$x$ * $y$) \% $m$}。
  \end{center}
    \pause
  有时模数过大或中间结果运算较复杂,均有可能出现即使没有乘法运算仍然整型溢出的现象。实践中,大量代码错误由未正确使用整数表示或出现整型溢出导致。
\end{frame}

\subsubsection{快速幂}

\begin{frame}
  \frametitle{求幂问题}
  不难理解为何到现在才将“求幂”视为一个问题:在整数中,如果不需要取模,若不使用更高精度的整数,除开 $0^n$ 与 $1^n$ 等平凡情况,即使是计算 $2^n$,$n$ 也不能超过 $64$ 或 $128$,否则结果将难以使用 \Cpp{} 原生类型表示,于是不需要担心 $\mathcal O (n)$ 算法的时间复杂度;而在浮点数中,有 \texttt{std::pow} 函数帮助完成更广泛的情况。
  
  \pause
  \emptyline
  在模意义下,求 $a^n \bmod m$ 是一个需要考虑如何快速计算的问题,其中 $a$ 可以是任意整数,而 $n$ 是非负整数。\\
  (当 $n = 0$ 时,认为 $a^0 = 1$ 对所有整数 $a$ 成立,包括 $0^0 = 1$。)
  
  \pause
  \emptyline
  后文中,将看到 ${\{ \langle n, a^n \rangle \}}_{n = 0}^{\infty}$ 可能含有相对复杂的结构,但对于求幂问题我们有简单方便的解决方案。
\end{frame}

\begin{frame}
  \frametitle{快速幂 -- 二进制表示}
  考虑 $n$ 的二进制表示 $n = \overline{n_k n_{k - 1} \cdots n_2 n_1 n_0}_{(2)} = \sum_{i = 0}^{k} n_i \cdot 2^i$。
  
  \pause
  根据公式 $a^{s + t} = a^s \cdot a^t$,又由于 $n_i$ 为 $0$ 或 $1$,我们有
  %
  \begin{align*}
    a^n &= a^{\sum_{i = 0}^{k} n_i \cdot 2^i} \\
    &= a^{n_0} \cdot a^{n_1 \cdot 2} \cdot a^{n_2 \cdot 4} \cdot \cdots \cdot a^{n_k \cdot 2^k} \\
    &= \prod_{n_i = 1} a^{2^i} \text{。}
  \end{align*}
  
  \pause
  \begin{example}
    由于 $13 = \overline{1101}_{(2)}$,有 $a^{13} = a \cdot a^4 \cdot a^8$。
    
    由于 $33289 = \overline{1000001000001001}_{(2)}$,有 $a^{33289} = a \cdot a^8 \cdot a^{512} \cdot a^{32768}$。
  \end{example}
\end{frame}

\begin{frame}
  \frametitle{快速幂 -- 反复平方}
  据此,结果可拆分成不超过 $\lceil \log_2 n \rceil$ 个形如 $a^{2^i}$ 的中间结果的乘积,而 $a^{2^i}$ 中的 $i$ 不超过 $\lfloor \log_2 n \rfloor$。
  
  \pause
  \emptyline
  对于 $a^{2^i}$,可以考虑反复平方:
  
  \begin{itemize}
    \item $a$ 平方后得到 $a^2 = a^{2^1}$。
    \item $a^2$ 平方后得到 $a^4 = a^{2^2}$。
    \item $a^4$ 平方后得到 $a^8 = a^{2^3}$。
    \item 以此类推,$a$ 反复平方 $i$ 次后恰好得到 $a^{2^i}$。
  \end{itemize}
  
  据此得到每个 $a^{2^i}$ 后,再使用 $n$ 的二进制表示,即可通过中间结果得到 $a^n \bmod m$。需要注意中间结果与最终计算时每一步都要取模。
  
  \pause
  假设求 $n$ 的二进制表示只需要 $\mathcal O (\log n)$ 的时间复杂度,容易看出快速幂的时间复杂度为 $\mathcal O (\log n)$。
\end{frame}

\begin{frame}[fragile]
  \frametitle{快速幂 -- \Cpp{} 代码实现}
  快速幂有很简便的代码实现方式。此处给出一种一边反复平方、一边计算结果的实现方式,此种方式的空间复杂度为 $\mathcal O (1)$。
  
\begin{minted}[fontsize = \footnotesize]{cpp}
int quick_pow(int a, int n) {
  int b = 1;
  for (; n; n >>= 1, a = (long long)a * a % M)
    if (n & 1)
      b = (long long)b * a % M;
  return b;
}
\end{minted}
  
  在第 $i$ 次进入循环体时,\texttt{a} 的值恰为 $a^{2^{i - 1}}$,而 \texttt{n \& 1} 恰为 $n_{i - 1}$。
\end{frame}

\begin{frame}
  \frametitle{快速幂 -- 其他应用}
  快速幂算法流程本身与数论并无太大关系,可以发现,只要是有结合律的运算由同一个对象连续进行大量次数,都可以使用快速幂在 $\mathcal O (\log n)$ 次运算内计算。
  
  快速幂的其他常见应用有:矩阵快速幂(取模或浮点数)、多项式快速幂(取模)等等。
  
  \pause
  \emptyline
  需要进行的运算次数关于 $n$ 的函数,是衡量快速幂算法效率的一个重要指标,在如矩阵快速幂等运算复杂度较高的场景中,运算次数需要尽可能少。可以发现,前文给出的快速幂算法使用不超过 $2 \log_2 n$ 次运算。针对特定的 $n$,找出运算次数尽可能少的算法,被称为最短加法链问题。容易发现,再短的加法链也至少需要 $\log_2 n$ 次运算,所以前文给出的快速幂算法在常数上最多劣一倍。
\end{frame}

\section{加法结构}

\begin{frame}
  \frametitle{未完成}
  此部分未完成,后续更新将在 \href{https://github.com/GitPinkRabbit/Number-Theory-in-Competitive-Programming}{算法竞赛中的数论 -- 系列课件 -- GitHub/GitPinkRabbit} 中上传。
\end{frame}

\section{乘法结构(上)}

\begin{frame}
  \frametitle{关于内容编排的说明}
  相比于剩余系的加法结构,剩余系的乘法结构则要难以把握得多。
  
  本系列课件分三节讨论剩余系的乘法结构,本课件中含有其中前两节,而第三节将包含于课件三《中国剩余定理》中。
\end{frame}

\subsection{简化剩余系与 Euler 函数}
\subsubsection{简化剩余系}

\begin{frame}
  \frametitle{简化剩余系}
  为了讨论剩余系的乘法结构,首先引入简化剩余系的概念。
  
  \begin{definition}[简化剩余系]
    在 $m$ 个剩余类中,拥有乘法逆元的剩余类的集合称为\fdf{模 $m$ 简化剩余系(缩系,reduced residue system modulo $m$)}。
    
    作为对比,模 $m$ 剩余系常被称为完全剩余系(简称完系)。
    
    记简化剩余系中的剩余类 $a$ 的逆元为 $a^{-1}$。
  \end{definition}
  
  \pause
  \emptyline
  根据课件一中的结论:只有与 $m$ 互素的整数才可以定义逆元(Bézout 定理),可以给出简化剩余系的另一个等价定义:
  
  \begin{itemize}
    \item $[0, m - 1]$ 中与 $m$ 互素的整数所在的剩余类组成了简化剩余系。
  \end{itemize}
  
  
\end{frame}

\begin{frame}
  \frametitle{简化剩余系 -- 例子}
  \begin{example}[简化剩余系]
    \begin{itemize}
      \item 当 $m = 4$ 时,缩系为 $\{ 1, 3 \}$,这是因为 $1 \cdot 1 = 3 \cdot 3 = 1$。
      \item 当 $m = 5$ 时,缩系为 $\{ 1, 2, 3, 4 \}$,这是因为 $1 \cdot 1 = 2 \cdot 3 = 3 \cdot 2 = 4 \cdot 4 = 1$,其中 $2, 3$ 互为逆元。
      \item 当 $m = 6$ 时,缩系为 $\{ 1, 5 \}$,其他剩余类 $0, 2, 3, 4$ 均没有乘法逆元。
      \item 当 $m = 1$ 时,缩系为 $\{ 0 \}$。这是唯一一个简化剩余系等于完全剩余系的例子。
    \end{itemize}
  \end{example}
  
  可以看出这些简化剩余系中的任意整数都与 $m$ 互素。
\end{frame}

\begin{frame}
  \frametitle{简化剩余系 -- 性质}
  由于逆元的存在性,简化剩余系有着很好的性质。
  
  \pause
  \begin{onlyenv}<2-6>
    \begin{itemize}
      \item 缩系对乘法封闭,即如果 $a, b$ 在缩系中,则 $a b$ 也在。这是因为 $(a b) (a^{-1} b^{-1}) = 1$。
        \pause
      \item 缩系对逆元封闭。$a$ 的逆元的逆元即为 $a$ 自身,这是由于 $a \cdot a^{-1} = a^{-1} \cdot a = 1$。
        \pause
      \item 上一条说明缩系中的剩余类要么两两配对,要么逆元为自身。例如,当 $m = 8$ 时,缩系 $\{ 1, 3, 5, 7 \}$ 中每一个的逆元均为自身。
        \pause
      \item 在缩系中可以定义除法,$a / b$ 可被定义为 $a \cdot b^{-1}$。
        \pause
      \item 但是,缩系不对加减法封闭。例如,当 $m = 9$ 时,$1 + 2 = 3$ 不在缩系中;除了 $m = 1$ 外,$0$ 均不在缩系中,而显然同一个数相减即可得到 $0$。
    \end{itemize}
  \end{onlyenv}
  
  \pause
  \begin{onlyenv}<7-10>
    \begin{itemize}
      \item 如果 $a$ 在缩系中,则 $-a$ 也在缩系中,因为 $(-a) \cdot (-a^{-1}) = 1$。
        \pause
      \item 缩系中任意剩余类 $a$ 均可以通过一次与缩系中的剩余类 $c$ 的乘法得到任意缩系中的剩余类 $b$,即 $a x = b$ 总是在缩系中有解,取 $x = b / a$ 即可。
        \pause
      \item 对于任意缩系中的剩余类 $c$,不存在两个不同的(完系中的)剩余类 $a, b$ 满足 $a c = b c$,即 $\{ 0, c, 2 c, \ldots, (m - 1) c \}$ 两两不同。$a c = b c \implies (a - b) c = 0 \implies a - b = 0 \cdot c^{-1} \implies a = b$。
        \pause
      \item 上一条说明取定剩余类 $c$ 后,映射 $a \mapsto a c$ 形成完系到自身的一个双射,即置换。换句话说,$\{ 0, c, 2 c, \ldots, (m - 1) c \}$ 仍然构成完系。同时,由于缩系对乘法封闭,也形成缩系的一个置换。
    \end{itemize}
  \end{onlyenv}
\end{frame}

\subsubsection{Euler 函数}

\begin{frame}
  \frametitle{Euler 函数 -- 定义}
  \begin{definition}[Euler 函数]
    模 $m$ 简化剩余系的大小称为 \fdf{$m$ 的 Euler 函数},记作 $\varphi(m)$。
  \end{definition}
  
  \pause
  一个等价的定义是,$\varphi(m)$ 为 $[0, m - 1]$ 中与 $m$ 互素的整数的个数。
  
  \pause
  \begin{example}[Euler 函数]
    \begin{mymulticols}[l][l]{3}
      \begin{itemize}
        \item $\varphi(1) = 1$
        \item $\varphi(2) = 1$
        \item $\varphi(3) = 2$
        \item $\varphi(4) = 2$
        \item $\varphi(5) = 4$
        \item $\varphi(6) = 2$
        \item $\varphi(7) = 6$
        \item $\varphi(90) = 24$
        \item $\varphi(315) = 144$
      \end{itemize}
    \end{mymulticols}
  \end{example}
  
  可以看出,除了 $m = 1, 2$ 外,$\varphi(m)$ 均为偶数,因为剩余类恰好通过 $a$ 与 $-a$ 两两配对(当 $m$ 为 $\ge 4$ 的偶数时,$m / 2$ 不在缩系中)。
\end{frame}

\begin{frame}
  \frametitle{Euler 函数 -- 性质}
  \begin{itemize}
    \item 当 $p$ 为素数时,\textboldcolor[orange]{$\varphi(p) = p - 1$}:\\
      除了 $0$ 不与 $p$ 互素外,$[1, p - 1]$ 中的整数均与 $p$ 互素。
      \pause
    \item 当 $p$ 为素数、$\alpha$ 为正整数时,\textboldcolor[orange]{$\varphi(p^\alpha) = (p - 1) \cdot p^{\alpha - 1}$}:\\
      与 $p^\alpha$ 互素当且仅当不为 $p$ 的倍数,显然 $[0, p^\alpha - 1]$ 中恰有 $p^{\alpha - 1}$ 个 $p$ 的倍数,扣除这些数即可。
      \pause
    \item 当整数 \alert{$a, b$ 互素}($a \perp b$)时,\textboldcolor[orange]{$\varphi(a b) = \varphi(a) \cdot \varphi(b)$},这个性质称为积性,即 Euler 函数是一个\fdf{积性函数}。\\
      此性质将在课件三中证明。
  \end{itemize}
\end{frame}

\begin{frame}
  \frametitle{Euler 函数 -- 计算}
  由前三个性质,可以推导出更多能够帮助我们进行计算的性质:
  
  \pause
  \begin{itemize}
    \item 设 $n$ 的标准分解式为 $n = p_1^{\alpha_1} p_2^{\alpha_2} \cdots p_k^{\alpha_k}$,则 \textboldcolor[orange]{$\displaystyle \varphi(n) = \prod_{i = 1}^{k} (p_i - 1) \cdot p_i^{\alpha_i - 1} = n \cdot \prod_{i = 1}^{k} \frac{p_i - 1}{p_i}$}:\\
      由 Euler 函数的积性,有 $\varphi(n) = \prod_{i = 1}^{k} \varphi(p_i^{\alpha_i})$,再由性质展开 $\varphi(p_i^{\alpha_i})$ 即可。
      \pause
    \item 当 $p$ 为素数、$i$ 为正整数,\textboldcolor[orange]{$\varphi(p \cdot i) = \begin{cases}
        (p - 1) \cdot \varphi(i) & \text{,$p \nmid i$} \\
        \phantom{(} p \phantom{{} - 1)} \cdot \varphi(i) & \text{,$p \mid i$}
      \end{cases}$}\\
      由上一条性质容易得到。
  \end{itemize}
  
  \pause
  可以根据这两条性质计算 Euler 函数:第一条指出使用标准分解式即可求值,第二条给出在筛法中求值的简便方法。
\end{frame}

\begin{frame}[fragile]
  \frametitle{Euler 函数 -- 使用标准分解式求值 -- \Cpp{} 代码}
  
  若使用标准分解式,只需修改素因数分解的试除法的代码:
  
\begin{minted}[fontsize = \scriptsize]{cpp}
int Euler_phi(int n) {
  int phi = n;
  for (int d = 2; d * d <= n; ++d)
    if (n % d == 0) {
      int p = d;
      phi = phi / p * (p - 1);
      while (n % p == 0)
        n /= p;
    }
  if (n != 1)
    phi = phi / n * (n - 1);
  return phi;
}
\end{minted}
  
  时间复杂度与素因数分解相同。
\end{frame}

\begin{frame}[fragile]
\frametitle{Euler 函数 -- 使用筛法求值 -- \Cpp{} 代码}
  
  若使用筛法,则可以一次性求出 $[1, n]$ 内所有数的 Euler 函数值:\\
  与 Euler 筛法相比,只需修改两处。
  
\begin{minted}[fontsize = \scriptsize]{cpp}
int Euler_phi[MaxN];

if (!is_composite[i]) {
  primes.push_back(i);
  Euler_phi[i] = i - 1;
}

if (i % p != 0) {
  Euler_phi[k] = (p - 1) * Euler_phi[i];
} else {
  Euler_phi[k] = p * Euler_phi[i];
  break;
}
\end{minted}
  
  时间复杂度为 $\mathcal O (n)$。
\end{frame}

\begin{structuregreen}

\begin{frame}
  \frametitle{Euler 函数 -- 例题 -- HAOI 2012《外星人》}
  \begin{block}{洛谷 P2350\ \ \href{https://www.luogu.com.cn/problem/P2350}{\color{blue!30!white}\footnotesize\faExternalLink}}
    以标准分解式的形式给出 $n = p_1^{\alpha_1} p_2^{\alpha_2} \cdots p_k^{\alpha_k}$,求出使得 $\varphi^{(x)}(n) = 1$ 成立的最小正整数 $x$。其中 $\varphi^{(x)}(n)$ 表示 $\varphi$ 嵌套 $x$ 次,即 $\varphi(\varphi( \cdots \varphi(n) \cdots ))$。
    
    \begin{itemize}
      \item $k \le 2000$,$p_i \le {10}^5$,$\alpha_i \le {10}^9$。
    \end{itemize}
  \end{block}
  
  样例:当 $n = 2^2 \cdot 3 = 12$ 时,$\varphi^{(3)}(12) = \varphi^{(2)}(4) = \varphi(2) = 1$。
  
  注意:$n$ 可能非常大,素因数的范围可以接受,但是幂次非常高。
  
  \pause
  \emptyline
  提示:从标准分解式的角度解释为什么当 $n \ge 3$ 时 $\varphi(n)$ 为偶数。
\end{frame}

\begin{frame}
  \frametitle{Euler 函数 -- 例题 -- HAOI 2012《外星人》-- 题解}
  当 $n \ge 3$ 时 $\varphi(n)$ 为偶数的原因是:每个奇素因数 $p$ 都会向 $\varphi$ 值中贡献一个 $p - 1$,为偶数,当 $n = 2^\alpha$ 没有奇素因数时,也必有 $\alpha \ge 2$ 导致 $\varphi(n) = 2^{\alpha - 1}$ 为偶数。
  
  \pause
  \emptyline
  这启发我们观察在迭代过程中,标准分解式中 $2$ 的幂次是如何变化的。只要 $n$ 为偶数,根据 $\frac{\varphi(n)}{n} = \prod \frac{p - 1}{p}$,可以看作删去了一个 $2$,但又通过 $\prod (p - 1)$ 添加了若干(也可能没有)$2$ 作为素因数。
  
  \pause
  \emptyline
  由于直到最后一步 $2 \to 1$ 时每一步都恰好如上删去一个 $2$,可以转而统计在过程中总共获得了多少个 $2$(\alert{也包括初始时已有的 $2$}),获得的个数即为迭代步数。\\
  当 $n$ 为奇数时,由于第一步迭代没有删去 $2$,故步数多一次。
\end{frame}

\begin{frame}
  \frametitle{Euler 函数 -- 例题 -- HAOI 2012《外星人》-- 题解}
  转而统计在过程中总共获得了多少个 $2$,获得的个数即为迭代步数。\\
  当 $n$ 为奇数时,由于第一步迭代没有删去 $2$,故步数多一次。
  
  \pause
  \emptyline
  由于每个素因数最终都必然被拆分为一个个 $2$,获得的 $2$ 的个数之和关于每个素因数独立。将通过 $n$ 能获得的 $2$ 的个数记作 $f(n)$,则有 $f(2) = 1$、$f(p) = f(p - 1)$、以及 $f(p \cdot i) = f(p) + f(i)$,据此可以 Euler 筛。
  
  \pause
  \emptyline
  答案为 $\displaystyle \left( \sum_{i = 1}^{k} f(p_i) \cdot \alpha_i \right) + [p_1 \ne 2]$。
\end{frame}

\begin{frame}
  \frametitle{Euler 函数 -- 例题 -- HAOI 2012《外星人》-- 回顾}
  Euler 函数的迭代在后文中还会见到。
  
  关于它的一个重要结论是对于所有偶数 $n$ 有 $\varphi(n) \le \frac{n}{2}$,并且对于所有奇数 $n$ 有 $\varphi(n)$ 为偶数。
  
  故最多第一次变为偶数后,每次迭代数值都减半,然后在 $\log_2 n$ 步内到达 $1$。换句话说,迭代步数 $< \log_2 n + 1$。
  
  \pause
  \emptyline
  事实上,我们有 $\log_3(n / 2) + 1 \le$ 迭代步数 $< \log_2 n + 1$。\\
  左侧在 $2 \cdot 3^\alpha$ 处取到等号。\\
  \pause
  若不存在更多的 Fermat 素数\footnote<3>{即 $2^{2^k} + 1$ 型素数},右侧可改为 $\le \log_2 n + (9 - \log_2 257) \approx \log_2 n + 0.994375$ 并在 $257$ 处取到等号。
\end{frame}

\end{structuregreen}

\subsection{Euler 定理与 Fermat 小定理}
\subsubsection{Euler 定理}

\begin{frame}
  \frametitle{Euler 定理}
  Euler 函数可不只有能表示缩系大小那么简单,接下来介绍 \fdf{Euler 定理(Euler's theorem)}。
  
  \pause
  \begin{theorem}[Euler 定理]
    对于任意正整数 $m$ 和任意与 $m$ \alert{互素}的整数 $a$,均有 $a^{\varphi(m)} \equiv 1$。\\
    使用剩余类的语言来说,对于任意模 $m$ \alert{简化剩余系}中的剩余类 $a$,均有 $a^{\varphi(m)} = 1$。
  \end{theorem}
  
  \pause
  \begin{example}[Euler 定理]
    \begin{mymulticols}[l][l]{2}
      \begin{itemize}
        \item $2^6 \equiv 1 \pmod{7}$
        \item $3^{40} \equiv 1 \pmod{100}$
        \item $5^{256} \equiv 1 \pmod{512}$
        \item $0^1 \equiv 1 \pmod{1}$
      \end{itemize}
    \end{mymulticols}
  \end{example}
\end{frame}

\begin{frame}
  \frametitle{Euler 定理 -- 证明}
  \begin{proof}[Euler 定理]
    前文中提到 $b \mapsto a b$ 形成缩系的一个置换,记缩系中全体元素为 $\{ b_1, b_2, \ldots, b_{\varphi(m)} \}$,则 $\{ a \cdot b_1, a \cdot b_2, \ldots, a \cdot b_{\varphi(m)} \}$ 恰好也形成缩系。求乘积得到 $\begin{aligned}[t] \textstyle \prod_{i = 1}^{\varphi(m)} b_i &= \textstyle \prod_{i = 1}^{\varphi(m)} a b_i \\ &= \textstyle a^{\varphi(m)} \prod_{i = 1}^{\varphi(m)} b_i \text{。} \end{aligned}$
    
    消去两侧的 $\prod_{i = 1}^{\varphi(m)} b_i$,即得 $a^{\varphi(m)} = 1$。
  \end{proof}
\end{frame}

\subsubsection{Fermat 小定理}

\begin{frame}
  \frametitle{Fermat 小定理}
  \begin{theorem}[Fermat 小定理]
    对于任意素数 $p$ 和任意不为 $p$ 的倍数的整数 $a$,均有 $a^{p - 1} \equiv 1$。\\
    使用剩余类的语言来说,对于任意非零剩余类 $a$,均有 $a^{p - 1} = 1$。
    
    \emptyline
    有时,Fermat 小定理也使用“对任意素数 $p$ 和任意整数 $a$,均有 $p \mid (a^p - a)$”这一形式表述。
  \end{theorem}
  
  \pause
  \begin{example}[Fermat 小定理]
    \begin{mymulticols}[l][l]{2}
      \begin{itemize}
        \item $2^6 \equiv 1 \pmod{7}$
        \item $10^{16} \equiv 1 \pmod{17}$
        \item $2^{22} \equiv 1 \pmod{23}$
        \item $1^{196} \equiv 1 \pmod{197}$
      \end{itemize}
    \end{mymulticols}
  \end{example}
\end{frame}

\begin{frame}
  \frametitle{Fermat 小定理 -- 证明}
  \begin{proof}[Fermat 小定理]
    对于 $a^{p - 1} \equiv 1$ 的形式,Euler 定义取 $m$ 为素数的特例即可。
    
    对于 $p \mid (a^p - a)$ 的形式,只需额外考虑 $a$ 为 $p$ 的倍数的情况。
  \end{proof}
  
  \pause
  可以发现,Fermat 小定理完全是 Euler 定理的特例。
  
  不过,Fermat 小定理的优势在于,不需要考虑求 $\varphi$ 值,但前提是模数为素数。
\end{frame}

\subsubsection{应用}

\begin{frame}
  \frametitle{高精度指数快速幂}
\end{frame}

\begin{frame}
  \frametitle{求逆元}
\end{frame}

\section{乘法结构(中)}

\begin{frame}
  \frametitle{未完成}
  此部分未完成,后续更新将在 \href{https://github.com/GitPinkRabbit/Number-Theory-in-Competitive-Programming}{算法竞赛中的数论 -- 系列课件 -- GitHub/GitPinkRabbit} 中上传。
\end{frame}

\section{附录}
\subsection*{附录}

\begin{frame}[c]{参考文献与致谢}
  \begin{itemize}
    \item OI Wiki,{\footnotesize\url{https://oi-wiki.org/}}
    \item Wikipedia,{\footnotesize\url{https://en.wikipedia.org/}}
    \item 初等数论学习笔记 I:同余相关,Alex\_Wei,{\footnotesize\url{https://www.cnblogs.com/alex-wei/p/Number_Theory.html}}
    \item 本课件编写时的哔哩哔哩直播间观众
  \end{itemize}
\end{frame}

\end{document}
